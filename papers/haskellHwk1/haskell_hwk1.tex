\title{Haskell Hwk1}
\author{
        Martin Kozeny\\
        CSCI 4501: Programming Language Structure\\
        Spring 2011
        University of New Orleans
}
\date{\today}




\documentclass[4pt]{article}
\usepackage{graphicx}
\usepackage{listings}
\lstnewenvironment{code}{\lstset{language=Haskell,basicstyle=\ttfamily\small}}{}
%\lstset{language=C,caption=Descriptive Caption Text,label=DescriptiveLabel}


\usepackage{listings}
\lstloadlanguages{Haskell}






\setlength{\hoffset}{-2.3cm} 
\setlength{\voffset}{-3.6cm}
\setlength{\textheight}{24.0cm} 
\setlength{\textwidth}{16cm}


\begin{document}


\maketitle


\section{Haskell code}

\lstinputlisting[language=Haskell]{files/Functions.hs}

\newpage
\section{Test script}
\lstinputlisting[language=Haskell]{files/Test_script.hs}
\newpage
\section{C code}
\begin{verbatim}
/*libraries*/
#include <ctype.h>
#include <math.h>


/*headers of functions*/
float triangleArea(float a, float v);
char changeCase(char a);
char * letterGrade(int grade);
int highestPowerDivisor(int number1, int number2);
int addRange(int a, int b, int c);
double areaTriangle(double a, double b, double c);
/*testing functions is in main thread*/
int main()
{
    /*TESTING OF TRIANGLE AREA*/
    printf("\r\nTESTING OF TRIANGLE AREA\r\n");
    float triangleA = triangleArea(5,3);
    if(triangleA != -1)
        printf ("1. Triangle area, where base is 5 and height is 3, is %f\r\n",triangleA);
    triangleA = triangleArea(4,2);
    if(triangleA != -1)
        printf ("1. Triangle area, where base is 4 and height is 2, is %f\r\n",triangleA);
    triangleA = triangleArea(7,8);
    if(triangleA != -1)
        printf ("1. Triangle area, where base is 7 and height is 8, is %f\r\n",triangleA);
    triangleA = triangleArea(14,5);
    if(triangleA != -1)
        printf ("1. Triangle area, where base is 14 and height is 5, is %f\r\n",triangleA);
    triangleA = triangleArea(-3,7);
    if(triangleA != -1)
        printf ("1. Triangle area, where base is -3 and height is 7, is %f\r\n",triangleA);
    
    /*TESTING OF CHANGE CASE*/
    printf("\r\nTESTING OF CHANGE CASE\r\n");
    printf ("2. Another case for char 'c' is '%c'\r\n",changeCase('c'));
    printf ("2. Another case for char 'A' is '%c'\r\n",changeCase('A'));
    printf ("2. Another case for char 'f' is '%c'\r\n",changeCase('f'));
    printf ("2. Another case for char 'G' is '%c'\r\n",changeCase('G'));
    printf ("2. Another case for char 'w' is '%c'\r\n",changeCase('w'));
    printf ("2. Another case for char 'X' is '%c'\r\n",changeCase('X'));
    
    /*TESTING OF LETTER GRADE*/
    printf("\r\nTESTING OF LETTER GRADE\r\n");
    printf ("3. Letter grade for 95 numeric grade is %s\r\n",letterGrade(95));
    printf ("3. Letter grade for 90 numeric grade is %s\r\n",letterGrade(90));
    printf ("3. Letter grade for 65 numeric grade is %s\r\n",letterGrade(65));
    printf ("3. Letter grade for 50 numeric grade is %s\r\n",letterGrade(50));
    printf ("3. Letter grade for 47 numeric grade is %s\r\n",letterGrade(47));
    printf ("3. Letter grade for 150 numeric grade is %s\r\n",letterGrade(150));
    printf ("3. Letter grade for -3 numeric grade is %s\r\n",letterGrade(-3));
    
    /*TESTING OF HIGHEST POWER DIVISOR*/
    printf("\r\nTESTING OF HIGHEST POWER DIVISOR\r\n");
    int highestPowerDiv = highestPowerDivisor(2,50);
    if(highestPowerDiv != -1)
        printf ("4. Highest power divisor for 2 and 50 is %i\r\n", highestPowerDiv);
    highestPowerDiv = highestPowerDivisor(5,250);
    if(highestPowerDiv != -1)
        printf ("4. Highest power divisor for 5 and 250 is %i\r\n", highestPowerDiv);
    highestPowerDiv = highestPowerDivisor(25,10);
    if(highestPowerDiv != -1)
        printf ("4. Highest power divisor for 25 and 10 is %i\r\n", highestPowerDiv);
    highestPowerDiv = highestPowerDivisor(7,100);
    if(highestPowerDiv != -1)
        printf ("4. Highest power divisor for 7 and 100 is %i\r\n", highestPowerDiv);
    highestPowerDiv = highestPowerDivisor(-2,13);
    if(highestPowerDiv != -1)
        printf ("4. Highest power divisor for -2 and 13 is %i\r\n", highestPowerDiv);
    
    /*TESTING OF ADD RANGE*/
    printf("\r\nTESTING OF ADD RANGE\r\n");
    int range=addRange(0, 5, 1);
    if(range != -1)
        printf ("5. Sum for range 0, 5, 1 is %i\r\n",range);
    range=addRange(1, 20, 3);
    if(range != -1)
        printf ("5. Sum for range 1, 20, 3 is %i\r\n",range);
    range=addRange(30, 20, 3);
    if(range != -1)
        printf ("5. Sum for range 30, 20, 3 is %i\r\n",range);
    range=addRange(20, 50, 4);
    if(range != -1)
        printf ("5. Sum for range 20, 50, 4 is %i\r\n",range);
    range=addRange(10, 20, -2);
    if(range != -1)
        printf ("5. Sum for range 10, 20, -2 is %i\r\n",range);
    
    /*TESTING OF AREA TRIANGLE*/
    printf("\r\nTESTING OF AREA TRIANGLE\r\n");
    double area = areaTriangle(3, 4, 6);
    if(area != -1)
        printf ("6. Triangle area for sides a=3, b=4 and c=6 is %f\r\n",area);
    area = areaTriangle(3, 4, 2);
    if(area != -1)
        printf ("6. Triangle area for sides a=3, b=4 and c=2 is %f\r\n",area);
    area = areaTriangle(2, 7, 8);
    if(area != -1)
        printf ("6. Triangle area for sides a=2, b=7 and c=8 is %f\r\n",area);
    area = areaTriangle(14, 15, 12);
    if(area != -1)
        printf ("6. Triangle area for sides a=14, b=15 and c=12 is %f\r\n",area);
    area = areaTriangle(7, 15, 7);
    if(area != -1)
        printf ("6. Triangle area for sides a=7, b=15 and c=7 is %f\r\n",area);
    return 0;
}
/*function 'triangleArea' count area of triangle clasiclly s= (a*v)/2
* input parameters are floats which are >0 (there is a control if parameters are > 0)
* and return value has also type Float*/
float triangleArea(float a, float v)
{
    if(a <= 0 || v <= 0) 
    {
        perror("1. Base and height have to be greater than 0");
        return -1;
    }
    return (a*v)/2;
}
/*function 'changeCase' change the letter case
* as an input char is expected and according to his case will be this letter changed to another case
* if is in lower case, than is changed to upper case,
* otherwise is expected as char in upper case and changed to lower case*/
char changeCase(char a)
{
    if(islower(a))
        return toupper(a);
    else
        return tolower(a);
}
/*function 'letterGrade' expected as input integer from 0 to 100
* returned value is String according to input number,
* if is lower than 0 or greater than 100, an error is thrown*/
char * letterGrade(int grade)
{
    if(grade >=90 && grade <=100)
        return "A";
    else if(grade >=80 && grade <=89)
        return "B";
    else if(grade >=70 && grade <=79)
        return "C";
    else if(grade >=60 && grade <=69)
        return "D";
    else if(grade >= 0 && grade <=59)
        return "E / F";
    else
    {
        perror("3. Bad numeric grade");
        return "does not exist";
    }
}
/*function 'highestPowerDivisor' expected as input two integers > 0
* (there is a control if parameters are > 0)
* are expected and returned value is also integer
* if first number modulo second number is nonzero, then the power is zero
* otherwise is first number modulo second number is zero, the power is at least 1
* so the result is 1 + and recursively count the power
* for first number and for second number divided by first number*/
int highestPowerDivisor(int number1, int number2)
{
    if(number1 <= 0 || number2 <= 0) 
    {
        perror("4. Input parameters must > 0");
        return -1;
    }
    if(number2 % number1 != 0)
        return 0;
    else
        return 1 + highestPowerDivisor(number1, number2/number1);
}
/*function 'addRange' expected as input input 3 integers are expected
* return value is also integer
* if first argument is greater than second argument
* or third argument lower than zero, an error is thrown
* if first argument plus third argument are higher than second argument,
* function returns first argument because second argument is border of range
* else if first argument plus third argument are equal to second argument,
* function returns first argument plus second argument
* (it is also possible return twice first argument plus third argument)
* else if first plus third argument is lower than second argument,
* we add first argument to recursion call where first argument is previous first
* argument * plus added range ( = third argument ), other parameters are same*/
int addRange(int a, int b, int c)
{
      int error= 0;
      if(a > b) 
      {
      perror("5. First parameter must be ≤ than second parameter");
      error=1;
      }
      if(c < 0)
      {
          perror("5. Third parameter must be > than 0");
      error=1;
      }
      if(error)
          return -1;
      if(a + c > b)
        return a;
      else if(a + c == b) 
        return a + b;
      else if(a + c < b)
        return a + addRange((a + c), b, c);
}
/*function 'areaTriangle' expects three Floats
* which are length of concrete side of triangle and returns value which is also float
* firstly it is tested if values of sides describe a trinagle
* if yes, area of triangle is counted using parameter 's' counted before*/
double areaTriangle(double a, double b, double c)
{
    if((a + b)<= c || (a + c)<=b || (b + c)<=a)
    {
        perror("6. Sides a, b, c do not describe a triangle");
        return -1;
    }
    double s= (a+b+c)/2;
    return sqrt(s*(s-a)*(s-b)*(s-c));
}
\end{verbatim}







\end{document}
